%************************************************
\chapter{Bayesian optimization for shaping the response of the stormwater systems}\label{ch:bayes}
%************************************************
Rapid advances in sensing, computation, and wireless communications are promising to merge the physical with the virtual.
Calls to build the ``smart'' city of the future are being embraced by decision makers.
While the onset of self-driving cars provides a good example that this vision is becoming a reality, the role  of information technology in the water sector has yet to be fleshed out.
These technologies stand to enable a leap in innovation in the distributed treatment of urban runoff, one of our largest environmental challenges. 

\

Retrofitting stormwater systems with sensors and controllers will allow the city to be controlled in real-time as a distributed treatment plant.
Unlike static infrastructure, which cannot adapt its operation to individual storms or changing land uses, ``smart'' stormwater systems will use system-level coordination to reduce flooding and maximize watershed pollutant removal.
Given the sheer number of stormwater control measures in United States, even a small improvement to their performance could lead to a substantial reduction in pollutant loads.
Intriguingly, such a vision is not limited by technology, which has matured to the point at which it can be ubiquitously deployed. 
Rather, the challenge is much more fundamental and rooted in a system-level understanding of environmental science.
Once stormwater systems become highly instrumented and controlled, how should they actually be operated to achieve desired watershed outcomes?
The answer to this question demands the development of a theoretical framework for smart stormwater systems. 
In this paper we lay out the requirements for such a theory.  Acknowledging that the broad adoption these systems may still be years away,  we also present and evaluate a modeling framework to allow for the simulation of smart stormwater systems before they become common place.
