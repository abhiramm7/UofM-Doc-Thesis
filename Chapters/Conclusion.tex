%************************************************
\chapter{Conclusion}\label{ch:conclusion}
%************************************************
\section{Summary of Discoveries}

The goal of this dissertation was to make fundamental discoveries that will inform the control of smart stormwater systems, specifically focusing on statistical learning approaches that can be used to generate safe and reliable control algorithms.
To that end, a number of fundamental discoveries were made in each chapter: 

\begin{itemize}
	\item \textbf{Chapter 2:} I have discovered that by controlling the hydrological responses in the stormwater basins the rate of removal of specific nutrients can be targeted. In this chapter I also develop a modular framework for simulating smart stormwater systems.
	\item \textbf{Chapter 3:} I have discovered that the response of the stormwater systems can be precisely shaped by relying on the sensor data. I demonstrate this by shaping the response of 4 $km^2$ watershed by coodinating the control actions of two assets to realize a watershed scale control objective.
	\item \textbf{Chapter 4:} I have demonstrated that deep reinforcement learning methods though show promise in the controlling stormwater systems, their performance is continent on the reward function, controller formulation, and choice of function approximator.
	\item \textbf{Chapter 5:} I have dicoved that flow control objectives in stormwater system can be realized with out real-time control and introducted a Bayesian-Optimization based algorithm 
	\item \textbf{Chapter 6:} I have introducted a python library for facilating the qualitative evalution of stormwater control algorithms. 
\end{itemize}

\section{Future Directions}


\noindent \textbf{Computer vision for hydrological monitoring}\\
Dynamics that govern the flow of water in urban systems are highly complex and are modeled using a parametrized version of simplified Navier-Stokes equations.
These model parameters require frequent calibration and validation to ensure that they accurately represent the physical system.
Given the steep financial investment and technical expertise required for maintaining a sensor network of flow and depth sensors, high-resolution data necessary for calibrating these models is rarely available.
Recent advances in electronics have made large scale deployment of cameras and edge processing financially viable, and computer vision methods have matured to a point where they can be reliably used for detection, segmentation, and tracking.
By adopting these methods for monitoring hydrological phenomenon, a camera can be repurposed to act as a flow, water level, and rainfall sensor.
Thus, creating an affordable alternative for acquiring high-resolution on-the-ground measurements.


\

\noindent \textbf{Uncertainty quantification for the control of urban water networks}\\
As the urban water infrastructure models become reliable, they can be leveraged to develop coordinated control strategies that enable infrastructure to reprogram itself in real-time to tackle dynamic weather conditions.
A major barrier for the adoption of such an approach, is the uncertainty associated with the underlying model, sensors measurements, and weather forecasts that dictates the actions taken by the control algorithm.
Quantifying these uncertainties will enable the decision makers to weigh the risks and rewards associated with control strategies and pick the one that benefits both the public and environment.
Bayesian optimization, with Gaussian Processes at its core, is an effective approach for searching though the solution space to identify an optimal control strategy and estimate the uncertainty associated with the identified strategy.
But the non-parametric nature of the Gaussian Processes limits the scalability of this approach.
By replacing the Gaussian Processes with Bayesian Neural Networks and Deep Gaussian processes, I want to extend the Bayesian Optimization approach for quantifying the uncertainties in large scale urban water networks. 
As a deliverable, I hope to create a open-source python toolbox that interfaces with the existing SWMM models to help the decision makers identify a control strategy and its associated uncertainty, for achieving a desired behavior in the urban water network.
