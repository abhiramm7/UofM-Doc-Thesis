%*******************************************************
% Abstract
%*******************************************************
%\renewcommand{\abstractname}{Abstract}
\pdfbookmark[1]{Abstract}{Abstract}
% \addcontentsline{toc}{chapter}{\tocEntry{Abstract}}
\begingroup
\let\clearpage\relax
\let\cleardoublepage\relax
\let\cleardoublepage\relax

\chapter*{Abstract}
% Basin intro to stormwater systems
Advances in wireless communication, electronics, and high performance computing is promising the merger of digital and phycisl relam.
The next of smarter stormwater systems, equipped with wireless sesnors and actuators, will have the ability to autonomously mitiage flooding and dynamically configure themself to achive watershed scale objectives.
Such a vision is not limited by tecnology, which has matured to a point where it can be deployed at scale, but rather by our understanding system scale control.
% Two lines summary of what this dissertation will talk about
This dissertation, leveraging statistical liearning methodologies, addresses these knowledge gaps and develops algorihtms and open-soutve tools for the autonomus control of stormwater systems.
% Into 2 lines on each chapter/
First chapter introductes a modular framework for simulaing stormater systems in which we cna interfaces the existing tools under a common framework for simulating real-time control in stoemater systesm.
Builinf on this framework, I demonstrate the effectiveness of control for ehcnaing nutienet removal in stormwater systesms.
Second chapter demonstrate the use of wireless sesnor actiator networkf for charection and coordniated control of a realworld watershed for aching system scale control objectives at the stomrmwater network's outlet.
Third chapter introduces a reinfocement learning based control alotihms for the automous control of stormwater systems and disucse the strengs and limitations of appliing these methdsol dot the control of a physical systems.
Fourth chapaper introductes a Bayesian Optimization based conaldothm for the control of sttetsm. 
Thsi demonstrate the first ever alfothms fot quantifun uncertantuy and connfing uncertanty.
In this final chalpter, I introce a python library for te quantifative evalution of sotemwatre control alothms.
Resuls from thsi bsisssetant stand to trnasition stowmater systems tinto ooreality.
%Short summary of the contents in English\dots a great guide by
%Kent Beck how to write good abstracts can be found here:
%\begin{center}
%\url{https://plg.uwaterloo.ca/~migod/research/beckOOPSLA.html}
%\end{center}

\endgroup

\vfill
