%*******************************************************
% Abstract
%*******************************************************
%\renewcommand{\abstractname}{Abstract}
\pdfbookmark[1]{Abstract}{Abstract}
% \addcontentsline{toc}{chapter}{\tocEntry{Abstract}}
\begingroup
\let\clearpage\relax
\let\cleardoublepage\relax
\let\cleardoublepage\relax

\chapter*{Abstract}
% Basin intro to stormwater systems and thesis goal
Rapid advances in wireless communication, embedded systems, and high-performance computing are promising a fusion of physical and digital water.
The next generation of stormwater systems --- equipped with wireless sensors and actuators --- will autonomously reconfigure themselves to prevent  flooding and achieve system scale objectives.
This vision of ``smart'' stormwater systems is not limited by technology, which has matured to the point where it can be ubiquitously deployed.
Instead, the challenge is much more fundamental and rooted in a system-level understanding of stormwater networks: \textit{once stormwater systems become highly instrumented, how should they be controlled to achieve the desired watershed outcomes?} This dissertation leverages statistical learning methods to begin closing fundamental knowledge gaps in the emerging field of smart water systems.
% Each chapter
% Second Chapter.
In the second chapter of this dissertation, we address the lack of simulation tools for developing control algorithms that target pollutants in stormwater runoff by introducing a new stormwater modeling framework.
Using this framework, we demonstrate real-time control’s potential for enhancing stormwater network’s nutrient removal efficiency. 
% Fourth Chapter 
In the third chapter, to characterize the controllability of a watershed, we carry out a real-world case study using a wireless sensor-actuator network to control the response of a watershed. %to achieve system-scale objectives.
Through this study, we learn that we can precisely shape the streamflows in a watershed by relying on the watershed's characterization identified from sensor measurements.
% Fourth Chapter 
In the fourth chapter, we investigate the use of a Reinforcement Learning-based control algorithm for synthesizing a stormwater controller that does not require explicit dynamical assumptions.
We demonstrate its effectiveness via a simulation-based evaluation and provide guidelines for adapting this approach for the control of stormwater systems.
% Fifth Chapter
In the fifth chapter, we introduce a Bayesian Optimization-based methodology for quantifying the uncertainties associated with control decisions and demonstrate its applicability using a simulation-based evaluation.
Quantifying these uncertainties enables us to identify the robust control strategies that can be implemented in a stormwater network.
% Sixth Chapter.
In the final chapter, we introduce an open-source Python library to facilitate the systematic quantitative evaluation of control algorithms.
This library provides a curated collection of stormwater control scenarios, coupled with an accessible programming interface and stormwater simulator, to provide a standalone package for developing stormwater control algorithms.
% Two lines on the future and conclude.
The discoveries made in this dissertation, along with the algorithms and tools developed, stand to provide the foundation for the development of a new generation of autonomous stormwater infrastructure.
\endgroup

\vfill
