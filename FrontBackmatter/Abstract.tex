%*******************************************************
% Abstract
%*******************************************************
%\renewcommand{\abstractname}{Abstract}
\pdfbookmark[1]{Abstract}{Abstract}
% \addcontentsline{toc}{chapter}{\tocEntry{Abstract}}
\begingroup
\let\clearpage\relax
\let\cleardoublepage\relax
\let\cleardoublepage\relax

\chapter*{Abstract}
\vspace{-0.5cm}
Rapid advances in wireless communication, embedded systems, and high-performance computing are promising the fusion of physical and digital water.
The next generation of stormwater systems --- equipped with wireless sensors and actuators --- will autonomously reconfigure themselves to prevent  flooding and achieve system scale objectives.
This vision of ``smart'' stormwater systems is not limited by technology, which has matured to the point where it can be ubiquitously deployed.
Instead, the challenge is much more fundamental and rooted in a system-level understanding of stormwater networks: \textit{once stormwater systems become highly instrumented, how should they be controlled to achieve the desired watershed outcomes?} This dissertation leverages statistical learning methods to begin closing fundamental knowledge gaps in the emerging field of smart water systems.
% Each chapter
% Second Chapter.
The second chapter of this dissertation addresses the lack of simulation tools for modeling pollutant interactions by introducing a new toolchain for coupling the existing hydraulic, hydrologic, and water quality models.
Using this toolchain, we demonstrate real-time control's potential for enhancing nutrient removal in a watershed.
% Real-time control's potential for enhancing nutrient removal in a watershed is demonstrated using this toolchain.
% Third Chapter 
In the third chapter, to characterize a watershed's controllability, a real-world case study is carried out using a wireless sensor-actuator network.
Through this study, the ability to precisely shape the hydrograph is quantified, illustrating the high level of granularity that can be achieved using real-time control. 
% Fourth Chapter 
Given that most state-of-the-art stormwater control algorithms require surrogate models or assume simplified dynamics, the fourth chapter introduces a Reinforcement Learning-based model-free algorithm for synthesizing stormwater controllers.
The efficacy of the algorithm is demonstrated via simulation, highlighting strong performance.
More importantly, a discussion is provided on the limitations of the approach, and a set of guidelines is presented for those seeking to apply Reinforcement Learning to stormwater control.
% Fifth Chapter
The fifth chapter in this dissertation introduces a Bayesian Optimization-based methodology for addressing the lack of knowledge relating to the uncertainty in stormwater control approaches and demonstrates its use for identifying robust control strategies.
% Sixth Chapter.
In the final chapter, an open-source Python library to facilitate the systematic quantitative evaluation of control algorithms is introduced.
This library provides a curated collection of stormwater control scenarios, coupled with an accessible programming interface and a stormwater simulator, to provide a standalone package for developing stormwater control algorithms.
% Two lines on the future and conclude.
The discoveries made in this dissertation, along with the algorithms and tools developed, seek to support the the development of a new generation of autonomous stormwater infrastructure.
%foundation for
\endgroup

\vfill

%The fifth chapter in this dissertation introduces a Bayesian Optimization-based methodology for quantifying the uncertainties associated with control decisions and demonstrate its use for identifying robust control strategies.
%Quantifying these uncertainties enables us to identify the robust control strategies that can be implemented in a stormwater network.
%We demonstrate its effectiveness via a simulation-based evaluation and provide guidelines for adapting this approach for the control of stormwater systems.

