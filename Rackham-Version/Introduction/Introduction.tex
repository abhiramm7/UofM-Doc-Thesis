%************************************************
\chapter{Introduction}\label{ch:introduction}
%************************************************
%What are stormwater systems and why are they needed?
As the cities grow larger and they alter the landscape of the watersheds.
As the watershed shape changes, it alters the how the water flows on the watershed.
This altered water flow can be dangerous and can be harmful to the people in the system.
Hence, we need to infrastructure in place to move this water away from the urban environment.
This is where, stormwater infrastructure comes in.
This infrastructure moves water accumulated in cities away from them into downstream water bodies.

\

% What are the problems with stormwater systems?
Stormwater systems though designed to handle the runoff, are not able to handle the rising demands in the urban environments. 
Redesigning and retrofitting these systems to handle the rising demands, such an approach might not work always. 
Given the dynamic nature of the systems, such a static solution might not always be the best way to tackle these challenges. 
Furthermore, given the these systems have to achieve multiple objectives, having them as static systems might not be the best solutions. 
Also these systems are designed as localized systems, with the intent that localized solutions will eventually scale up and improve the performance of the system as a whole.
But Maryland et al have demonstrated that might not always be the case. 
Hence, we need a more system scale approach that takes into account all the moving parts and treats the entire network as a single entity. 

\

% How are people trying to solve it?
These rising demands can be addressed by rebuilding the infrastructure with larger capacities to handle the increasing inflows. 
Though adding capacity to the existing system, can help us handle the incoming flows, it is an expensive process and it is not sure that it will work.
Alternatively, by retrofitting the existing system with sensors and controllers, we can monitor the state of the stormwater network in real-time and control its response to achieve system scale objectives.
This enables us to dynamically control the stormwater network to fully utilize the existing potential in the stormwater network to handle the incoming stormevents. 
Dynamics of stormwater networks are inherently complex and given the scale and the impact they can have on general public, we need good algorithms for controlling them.
Over the past decade, there has been a lot of interest on developing algorithms for the control of stormwater systems.


\ 

\section{Control of Stormwater Systems} 
% What are the different approaches being used for controlling stormwater systems?
The state-of-the-art in stormwater control can be broadly classified under two categories, based on how they identify control actions:
\begin{itemize}
	\item Control algorithms reliant on parametrized models (e.g.\ Model Predictive Control) for identifying control actions.
	\item Search based algorithms (e.g.\ evolutionary approaches like Genetic Algorithms) that exhaustively simulate models for identifying control actions.
	\item Heuristic based approaches that identify the control actions solely based on the state of the system. (e.g. Fuzzy logic controllers). 
\end{itemize}
Though these control algorithms have been applied for localized control in stormwater systems\footnote{e.g.\ maintaining constant water levels and flows in individual basins.}, their investigation in the context of coordinated control has been limited.
To fully realize the potential of the stormwater infrastructure and to safeguard our water bodies, we need to synthesize control algorithms that are able to coordinate the response of many distributed control assets in the network, while simultaneously achieving a diverse set of water quality and flow objectives. 
Technologically, we are at a point where we can monitor and control these assets in real-time, but the development of control algorithms is hampered by a number of fundamental knowledge gaps.

\


\graffito{MPC in stormwater control literature is broadly. In this dissertation, MPC is the explicit use of process based dynamical models for control.}
\

% What are the limitations of the existing approaches?
% Also why you cannot use resorvior optimization based approaches.

\

% What is need for your defense ?
Our existing stormwater infrastructure systems are unable to keep pace with rapidly evolving storm events and changing landscapes.
These infrastructure systems --- designed for an ``average'' event --- are still proving to be inefficient in tackling dynamic weather conditions and achieving diverse urban sustainability objectives\footnote{e.g.\ improving water quality and minimizing erosion}.
While existing stormwater systems could be rebuilt to reduce flooding and improve water quality, such an undertaking is often not financially viable, nor guaranteed to work.
In lieu of new construction, one alternative would be to retrofit  existing stormwater systems with sensors and controllers, so that these systems can be dynamically controlled in real-time to achieve the desired objectives.
The goal of my dissertation is to make fundamental discoveries that will inform the control of smart stormwater systems, specifically focusing on statistical learning approaches that can be used to generate safe and reliable control algorithms.

%What is the need for new generation of smart stormwater systems?
\section{Statistical methods}

% What are knowledge gaps preventing us from building such approaches. 
\subsection*{Knowledge Gaps}
\begin{enumerate}
	\item We do not know how to design control algorithms that can target pollutants in stormwater runoff, nor do we have the simulation tools necessary for such studies.
	\item We do not know to how to characterize the controllability of an urban watershed, especially in the context of water quality.
	\item We do yet know how to synthesize control algorithms for distributed storm-water assets without making explicit dynamical assumptions (e.g.\ linearity).
	\item We do not know how to quantify the uncertainty of algorithms used in the real-time control of stormwater systems.
	\item We do not know how to explicitly incorporate and account for hydraulic travel time within a real-time controlled system.
	\item We do not have open platforms for the systematic evaluation and comparison of different control algorithms.
\end{enumerate}


%What is the goal of your dissertation?
\section{Dissertation Outline}
% Describe your dissertation and then seaway into the chapter descriptions.
My dissertation addresses these knowledge gaps, leveraging statistical approaches, to develop tools and algorithms for enabling control of stormwater systems. 
The first chapter of this dissertation focuses on the development of a theoretical framework and the necessary tools for simulating control in stormwater systems. 
The second chapter demonstrates how a real-world wireless sensor network can be used for shaping the flow response of an entire urban watershed. 
In the next three chapters, various control algorithms are proposed, and their performance is evaluated across diverse scenarios to quantify the strengths and limitations.
In the final chapter, I introduce a python-based simulation sandbox, which is being developed specifically for the systematic evaluation and comparison of stormwater control algorithms.

